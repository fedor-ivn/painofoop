% (The MIT License)
%
% Copyright (c) 2023 Yegor Bugayenko
%
% Permission is hereby granted, free of charge, to any person obtaining a copy
% of this software and associated documentation files (the 'Software'), to deal
% in the Software without restriction, including without limitation the rights
% to use, copy, modify, merge, publish, distribute, sublicense, and/or sell
% copies of the Software, and to permit persons to whom the Software is
% furnished to do so, subject to the following conditions:
%
% The above copyright notice and this permission notice shall be included in all
% copies or substantial portions of the Software.
%
% THE SOFTWARE IS PROVIDED 'AS IS', WITHOUT WARRANTY OF ANY KIND, EXPRESS OR
% IMPLIED, INCLUDING BUT NOT LIMITED TO THE WARRANTIES OF MERCHANTABILITY,
% FITNESS FOR A PARTICULAR PURPOSE AND NONINFRINGEMENT. IN NO EVENT SHALL THE
% AUTHORS OR COPYRIGHT HOLDERS BE LIABLE FOR ANY CLAIM, DAMAGES OR OTHER
% LIABILITY, WHETHER IN AN ACTION OF CONTRACT, TORT OR OTHERWISE, ARISING FROM,
% OUT OF OR IN CONNECTION WITH THE SOFTWARE OR THE USE OR OTHER DEALINGS IN THE
% SOFTWARE.

\documentclass[nobrand,anonymous,nodate,nosecurity]{huawei}
\usepackage{enumerate}
\usepackage{multicol}
\usepackage{href-ul}
\usepackage{ffcode}
\begin{document}

{\sffamily{\bfseries\Large Pain of OOP}\\
Series of lectures by \href{https://www.yegor256.com}{Yegor Bugayenko} presented
to students of \href{https://innopolis.university/en/}{Innopolis University} in 2023.\\
% and \href{https://www.youtube.com/playlist?list=PLaIsQH4uc08woJKRAA7mmjs9fU0jeKjjM}{video recorded}}

% The entire set of slide decks is in \href{https://github.com/yegor256/ssd16}{yegor256/ssd16} GitHub repository.

\begin{abstract}
The course is a critical review of the current situation in object-oriented programming,
especially in Java, C++, Ruby, and JavaScript worlds. At the course, certain programming
idioms, which sometimes are called ``best practices,'' are criticized for their
negative impact on code quality, including static methods, NULL references, getters
and setters, ORM and DTO, annotations, traits and mixins, inheritance, and many others.
Much ``cleaner'' object-oriented programming practices will be proposed instead. Most
lectures are organized as reviews of existing code snippets from well-known
open source libraries.
\end{abstract}

% \section*{Introduction}

\textbf{What is the goal?}\\
The primary objective of the course is to help students understand the
difference between ``object thinking,'' originally motivated the
appearance of OOP, and the modern practices that often
severly impact the quality of code in a negative way.

\textbf{Who is the teacher?}\\
Yegor is developing software for more than 30 years, being a hands-on programmer
(see his GitHub account: \href{https://github.com/yegor256}{@yegor256})
and a manager of other programmers. At the moment Yegor is a director
of an R\&D laboratory in Huawei. His primary research focus is
software quality problems. Some of the lectures he has recently presented
at some software conferences could be found at
\href{https://www.youtube.com/channel/UCr9qCdqXLm2SU0BIs6d_68Q}{his YouTube channel}.
Yegor also published a \href{https://www.yegor256.com/books.html}{few books}
and wrote a \href{https://www.yegor256.com/contents.html}{blog} about software engineering
and object-oriented programming.
Yegor previously tought a few courses in
Innopolis University (Kazan, Russia)
and HSE University (Moscow, Russia),
for example,
\href{https://github.com/yegor256/ssd16}{Software Systems Design (2021)},
\href{https://github.com/yegor256/eqsp}{Ensuring Quality in Software Projects (2022)},
and
\href{https://github.com/yegor256/ppa}{Practical Program Analysis (2023)}
(all videos are available).

\textbf{Why this course?}\\
Maintainability of object-oriented software that most of us programmers write these days is
way below our expectations. Two main reasons for that is our misunderstanding
of what objects are. This course may help clear things up.

\textbf{What's the methodology?}\\
Each lecture is an overview of existing source code from well-known open source
libraries and frameworks, mostly from Apache. In an interactive discussion mode
we will go through their code and find out whether their programming practices
are in line with ``object thinking'' .

\newpage
\section*{Course Structure}

Prerequisites to the course (it is expected that a student knows this):

\begin{itemize}
\item How to write code
\item How to design software
\end{itemize}

After the course a student \emph{hopefully} will understand:

\begin{itemize}
\item What is the difference between objects and data?
\item Why static methods are bad?
\item What is immutability and why it's good?
\item How to design a constructor?
\item How to handle exceptions right?
\item What data hiding is for?
\item What's wrong with Printers, Writers, Scanners, and Readers?
\item Why NULL references are a billion-dollar mistake?
\item What's wrong with mixins and traits?
\item How to avoid getters and setters?
\item How declarative programming is better than imperative?
\item Why composition is better than inheritance?
\item Where to store the data if DTO is a bad practice?
\item How to avoid ORM design pattern?
\item Why long names of variables is bad design?
\item Why type casting and type checking are against OOP?
\item What is cohesion and why it matters?
\item How to apply SOLID and SRP principles?
\item What Inversion of Control is for and why DI Containers are evil?
\item Why MVC is a bad design idea?
\end{itemize}

\newpage
\section*{Lectures}

The following topics are discussed:

\newlist{lectures}{enumerate}{10}
\setlist[lectures]{label*=\arabic*.}
\begin{lectures}
\item Algorithms
\item Static Methods
\item Getters
\item Setters
\item ``-er'' suffix
\item NULL references
\item Type casting and reflection
\item Inheritance
\end{lectures}

\newpage
\section*{Grading}

\texttt{[45\%]} \\
Students form groups of 1--3 people. Each group presents
its own public GitHub repository with a software module inside.
It may be any piece of object-oriented software as long as it has more than 5,000 lines
of code written personally by the students.
Higher grades will be given
for better object-oriented practices used in the code. Lower grades will be
given for the precense of static methods, NULLs, getters, setters, mutability,
ORM, DTO, factories, type casting, and other anit-OOP practices (as suggested during the course).

\texttt{[30\%]} \\
At the laboratory classes each group will have to complete three
home works and defend them verbally on-site.

\texttt{[25\%]} \\
The attendance is tracked at the lectures. If more than 75\% of lectures were attended,
the attendance counts (otherwise ignored).

\newpage
\section*{Learning Material}

The following books are highly recommended to read (in no particular order):

\begin{multicols}{2}\small\raggedright
{David West}, \emph{Object Thinking}, 2004\\[3pt]
{David West}, \emph{Design Thinking}, 2017\\[3pt]
{Robert Martin}, \emph{Clean Code}, 2008\\[3pt]
{Steve McConnell}, \emph{Code Complete}, 1993\\[3pt]
{Yegor Bugayenko}, \emph{Elegant Objects}, 2016\\[3pt]
Blog posts of Yegor Bugayenko, \href{https://www.yegor256.com/tag/oop}{on his blog}\\[3pt]
Video lectures of Yegor Bugayenko \href{https://www.youtube.com/playlist?list=PLaIsQH4uc08yw2CsNv5OV30GfKE6XVGii}{on YouTube}\\[3pt]
Object Thinking meetup presentations, \href{https://www.youtube.com/watch?v=yT6oO28wEik&list=PLaIsQH4uc08yetzX86w1pPck1QtGEy_ik}{on YouTube}\\[3pt]
\end{multicols}

\end{document}
